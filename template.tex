\documentclass[12pt]{article}
\usepackage[utf8]{inputenc}
\usepackage{geometry}
 

\usepackage[lining]{ebgaramond}

\usepackage{lipsum}


\usepackage{amsmath}
\usepackage{cases}


\title{Test for \LaTeX~Environment}
\author{Yaojin Sun}
\date{\today}
 
\begin{document}
\maketitle

\begin{abstract}
	This is a basic document for testing \LaTeX~environment.
\end{abstract}

\section{Font}
This document is a sample document to 
test font families and font typefaces.
\\
\verb|qcr|
{\fontfamily{qcr}\selectfont
This text uses a different font typeface 0123456789 
}\\
\verb|cmr|
{\fontfamily{cmr}\selectfont
This text uses a different font typeface 0123456789
}\\
\verb|lmr|
{\fontfamily{lmr}\selectfont
This text uses a different font typeface 0123456789
}\\
\verb|lmdh|
{\fontfamily{lmdh}\selectfont
This text uses a different font typeface 0123456789
}\\
\verb|qtm|
{\fontfamily{qtm}\selectfont
This text uses a different font typeface 0123456789
}\\
\verb|qpl|
{\fontfamily{qpl}\selectfont
This text uses a different font typeface 0123456789
}\\
\verb|qbk|
{\fontfamily{qbk}\selectfont
This text uses a different font typeface 0123456789
}\\
\verb|qcs|
{\fontfamily{qcs}\selectfont
This text uses a different font typeface 0123456789
}\\
\verb|ptm|
{\fontfamily{ptm}\selectfont
This text uses a different font typeface 0123456789
}\\
\verb|lmtt|
{\fontfamily{lmtt}\selectfont
This text uses a different font typeface 0123456789
}\\

\subsection{Garamond}
Garamond is my favorate font compared with classic fonts embeded in \LaTeX.\\

\verb|\usepackage[lining]{ebgaramond}|
This is 

\section{Footnote}
Footnotes\footnote{A footnote is a note
of reference, explanation, or comment that is
usually placed below the text on a printed page.}
can be a nuisance. This is especially true if
there are many.\footnote{Like here.} The more you see
them, the more annoying they get.\footnote{Got it?}



\section{Quote}
Blah blah blah blah blah blah blah blah blah blah blah.
\begin{quote}
Next to the originator of a good sentence
is the first quoter of it. \\
\emph{Ralph Waldo Emerson}
\end{quote}
Blah blah blah blah blah blah blah blah.

\section{Center}
\begin{center}
Blah.\\
Blah blah blah.
Blah blah blah blah blah\\
blah blah blah blah blah
blah blah blah blah blah.
\end{center}

\section{Tabular}
\begin{tabular}{ |p{3cm}|p{3cm}|p{3cm}|  }
\hline
\multicolumn{3}{|c|}{Country List} \\
\hline
Country Name     or Area Name& ISO ALPHA 2 Code &ISO ALPHA 3 \\
\hline
Afghanistan & AF &AFG \\
Aland Islands & AX   & ALA \\
Albania &AL & ALB \\
Algeria    &DZ & DZA \\
American Samoa & AS & ASM \\
Andorra & AD & AND   \\
Angola & AO & AGO \\
\hline
\end{tabular}

\begin{table}[h!]
\centering
\begin{tabular}{c c c c} 
 \hline
 Col1 & Col2 & Col2 & Col3 \\ 
 \hline
 1 & 6 		& 87837 	& 787 \\ 
 2 & 7 		& 78 		& 5415 \\
 3 & 545 	& 778 		& 7507 \\
 4 & 545 	& 18744 	& 7560 \\
 5 & 88 	& 788 		& 6344 \\ 
 \hline
\end{tabular}
\caption{Table to test captions and labels(Actually, this is one of most used style for academic published result.)}
\label{table:1}
\end{table}


\section{Itemize}

\begin{itemize}
	\item First item.
	\item Second item. Text works as usual here.
	\item Third item is a list. Different labels here.
	\begin{itemize}
		\item First nested item.
		\item Second item.
	\end{itemize}
\end{itemize}




\section{Math}

\subsection{Inline Formula}
Here is an inline formula:
$   V = \frac{4 \pi r^3}{3}  $.

\subsection{Displayed Formula}
And appearing immediately below
is a displayed formula:
$$  V = \frac{4 \pi r^3}{3}  $$

\subsection{Block}
Joined Hypothesis:
\begin{equation}
H_0:\forall a_i=0;\\
H_a: \exists a_i \neq 0;\\
\end{equation}

\subsection{Use Package}
\verb|\usepackage{amsmath}|
\subsubsection{Align}
One Example:
\begin{align*} 
2x - 5y &=  8 \\ 
3x + 9y &=  -12
\end{align*}
Another example:
\begin{align*}
x&=y           &  w &=z              &  a&=b+c\\
2x&=-y         &  3w&=\frac{1}{2}z   &  a&=b\\
-4 + 5x&=2+y   &  w+2&=-1+w          &  ab&=cb
\end{align*}

\subsubsection{Split}
\begin{equation} \label{eq1}
\begin{split}
A & = \frac{\pi r^2}{2} \\
 & = \frac{1}{2} \pi r^2
\end{split}
\end{equation}


\subsubsection{Multline}
\begin{multline*}
p(x) = 3x^6 + 14x^5y + 590x^4y^2 + 19x^3y^3\\ 
- 12x^2y^4 - 12xy^5 + 2y^6 - a^3b^3
\end{multline*}

\subsubsection{Subequations}
\begin{subequations}
Maxwell's equations:
\begin{align}
        B'&=-\nabla \times E,\\
        E'&=\nabla \times B - 4\pi j,
\end{align}
\end{subequations}

\subsection{Use of Symbols}
It is so convenient to use \verb|\[| and \verb|\]| to block one equation.
\[
 A \overset{!}{=} B; A \stackrel{!}{=} B
\]
\subsubsection{Brace}
\[
 z = \overbrace{
   \underbrace{x}_\text{real} + i
   \underbrace{y}_\text{imaginary}
  }^\text{complex number}
\]

\subsection{Box}
\begin{equation}
 \boxed{x^2+y^2 = z^2}
\end{equation}

\subsection{Label}
\begin{equation} \label{eq:someequation}
5^2 - 5 = 20
\end{equation}

this references the equation \ref{eq:someequation}.

\subsection{Case}
\verb|\usepackage{cases}|
\begin{numcases}{|x|=}
x, & for $x \geq 0$\\
-x, & for $x < 0$
\end{numcases}



\end{document}