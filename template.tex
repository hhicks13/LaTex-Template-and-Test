\documentclass[12pt]{article}
\usepackage[utf8]{inputenc}
\usepackage{geometry}
 

\usepackage[lining]{ebgaramond}

\usepackage{lipsum}


\usepackage{amsmath}
\usepackage{cases}


\title{Test for \LaTeX~Environment}
\author{Yaojin Sun}
\date{\today}
 
\begin{document}
\maketitle

\begin{abstract}
	This is a basic document for testing \LaTeX~environment.
\end{abstract}

\section{Font}
This document is a sample document to 
test font families and font typefaces.
\\
\verb|qcr|
{\fontfamily{qcr}\selectfont
This text uses a different font typeface 0123456789 
}\\
\verb|cmr|
{\fontfamily{cmr}\selectfont
This text uses a different font typeface 0123456789
}\\
\verb|lmr|
{\fontfamily{lmr}\selectfont
This text uses a different font typeface 0123456789
}\\
\verb|lmdh|
{\fontfamily{lmdh}\selectfont
This text uses a different font typeface 0123456789
}\\
\verb|qtm|
{\fontfamily{qtm}\selectfont
This text uses a different font typeface 0123456789
}\\
\verb|qpl|
{\fontfamily{qpl}\selectfont
This text uses a different font typeface 0123456789
}\\
\verb|qbk|
{\fontfamily{qbk}\selectfont
This text uses a different font typeface 0123456789
}\\
\verb|qcs|
{\fontfamily{qcs}\selectfont
This text uses a different font typeface 0123456789
}\\
\verb|ptm|
{\fontfamily{ptm}\selectfont
This text uses a different font typeface 0123456789
}\\
\verb|lmtt|
{\fontfamily{lmtt}\selectfont
This text uses a different font typeface 0123456789
}\\

\subsection{Garamond}
Garamond is my favorate font compared with classic fonts embeded in \LaTeX.\\

\verb|\usepackage[lining]{ebgaramond}|
This is 

\section{Footnote}
Footnotes\footnote{A footnote is a note
of reference, explanation, or comment that is
usually placed below the text on a printed page.}
can be a nuisance. This is especially true if
there are many.\footnote{Like here.} The more you see
them, the more annoying they get.\footnote{Got it?}



\section{Quote}
Blah blah blah blah blah blah blah blah blah blah blah.
\begin{quote}
Next to the originator of a good sentence
is the first quoter of it. \\
\emph{Ralph Waldo Emerson}
\end{quote}
Blah blah blah blah blah blah blah blah.

\section{Center}
\begin{center}
Blah.\\
Blah blah blah.
Blah blah blah blah blah\\
blah blah blah blah blah
blah blah blah blah blah.
\end{center}

\section{Tabular}
\begin{tabular}{ |p{3cm}|p{3cm}|p{3cm}|  }
\hline
\multicolumn{3}{|c|}{Country List} \\
\hline
Country Name     or Area Name& ISO ALPHA 2 Code &ISO ALPHA 3 \\
\hline
Afghanistan & AF &AFG \\
Aland Islands & AX   & ALA \\
Albania &AL & ALB \\
Algeria    &DZ & DZA \\
American Samoa & AS & ASM \\
Andorra & AD & AND   \\
Angola & AO & AGO \\
\hline
\end{tabular}

\begin{table}[h!]
\centering
\begin{tabular}{c c c c} 
 \hline
 Col1 & Col2 & Col2 & Col3 \\ 
 \hline
 1 & 6 		& 87837 	& 787 \\ 
 2 & 7 		& 78 		& 5415 \\
 3 & 545 	& 778 		& 7507 \\
 4 & 545 	& 18744 	& 7560 \\
 5 & 88 	& 788 		& 6344 \\ 
 \hline
\end{tabular}
\caption{Table to test captions and labels(Actually, this is one of most used style for academic published result.)}
\label{table:1}
\end{table}


\section{Itemize}

\begin{itemize}
	\item First item.
	\item Second item. Text works as usual here.
	\item Third item is a list. Different labels here.
	\begin{itemize}
		\item First nested item.
		\item Second item.
	\end{itemize}
\end{itemize}








\end{document}